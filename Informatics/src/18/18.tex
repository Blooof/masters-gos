\documentclass[a4paper,12pt]{article}
\usepackage[russian]{babel}
\usepackage[utf8]{inputenc}
\usepackage[hidelinks]{hyperref}
\usepackage{amssymb}
\usepackage{amsmath}

\newcommand{\MExpect}{\mathsf{M}}

\begin{document}
\section*{18. Взаимная информация и ее свойства. Примеры вычисления. Информационная емкость и пропускная способность. Теоремы кодирования. Симметричные каналы. Пропускная способность гауссовского канала.}
\url{http://books.ifmo.ru/file/pdf/723.pdf}

Для заданного произведения $XY=\{(x,y),p(x,y)\}$ дискретных ансамблей $X$ и $Y$ нужно количественно измерить информацию об элементах $x \in X$ входного ансамбля, содержащуюся в выходных символах $y \in Y$. Подходящей мерой такой информации является взаимная информация, определяемая для любых пар $(x,y) \in{XY}$ соотношением $I(x;y)=I(x)-I(x|y)$ (где $I(x)=-\log{p(x)}$ -- собственная информация, $H(X)=\MExpect[-\log p(x)]=-\sum_{x\in X}{p(x)\log p(x)}$ -- энтропия, $I(x|y)=-\log p(x|y)$ -- условная собственная информация).

Уменьшаемое $I(x)$ представляет собой количество собственной информации, содержащейся в сообщении $x$. Вычитаемое -- условная собственная информация при известном $y$, иными словами, это количество информации, оставшейся в $x$ после получения $y$. Разность $I(x;y)$ представляет собой изменение информации в $x$ благодаря получению $y$.

Свойства взаимной информации:
\begin{enumerate}
\item Симметричность: $I(x;y)=I(y;x)$.
\item Если $x$ и $y$ независимы, то $I(x;y)=0$.
\end{enumerate}

\textit{Средней взаимной информацией} между ансамблями $X$ и $Y$ называется величина $I(X;Y)=\MExpect [I(x;y)]$, где $I(X;Y)=\sum_{x\in X}\sum_{y\in Y}p(x,y)\log\frac{p(x|y)}{p(y)}$.

Свойства:
\begin{enumerate}
\item Симметричность $I(X;Y)=I(Y;X)$
\item Неотрицательность $I(X;Y) \ge 0$
\item Тождество $I(X;Y)=0$ тогда и только тогда, когда ансамбли $X$ и $Y$ независимы.
\item $I(X;Y)=H(X)-H(X|Y)=H(Y)-H(Y|X)=H(X)+H(Y)-H(XY)$
\item $I(X;Y)\le\min\{H(X),H(Y)\}$
\item $I(X;Y)\le\min\{\log|X|,\log|Y|\}$
\item $I(X;Y)$ -- выпуклая вверх функция $p(x)$
\item $I(X;Y)$ -- выпуклая вних функция $p(y|x)$
\end{enumerate}

\subsection*{Информационная емкость и пропускная способность}
Рассмотрим дискретный стационарный канал. Мы знаем, что количество информации о входных символах $X$ канала, содержащееся в выходных символах $Y$ определяется средней взаимной информацией $I(X;Y)$. Это верно при передаче одного символа канала. При использовании кодов длины $n$ количество информации, получаемой декодером при передаче доного слова в среднем составит $I(X^n;Y^n)$ бит, что соответствует скорости передачи информации $\frac{1}{n}I(X^n;Y^n)$ бит/символ канала. Эта величина зависит от переходных вероятностей $\{p(y|x), y\in Y^n,x\in X^n\}$ и от распределения вероятностей на входе канала $\{p(x),x\in X^n\}$. Результирующая скорость получается $\max\limits_{\{p(x)\}}\frac{1}{n}I(X^n;Y^n)$ бит/символ канала. Величина $C_0=\sup\limits_n\max\limits_{\{p(x)\}}\frac{1}{n}I(X^n;Y^n)$ называется \textit{информационной емкостью} канала.

Допустим, у нас есть код канала длины $n$, количеством последовательностей $M$ (\textit{мощность} кода), $R=\frac{\log M}{n}$ -- скорость кода. Число $C$ называется \textit{пропускной способностью канала}, если при любой скорости кода $R<C$ существуют коды, обеспечивающие сколь угодно малую вероятность ошибки и, напротив, при $R>C$ существует константа $\varepsilon>0$, что вероятность ошибки любого кода ограничена снизу величиной $\varepsilon$.

Обратная теорема кодирования утверждает, что информационная ёмкость $C_0$ ограничивает сверхху скорость, при которой достижима сколь угодно малая вероятность ошибки. Прямая теорема кодирования утверждает, что при скорости, сколь угодно близкой к $C_0$ достижима сколь угодно малая вероятность ошибки. 

\subsection*{Теоремы кодирования}

\textbf{Обратная теорема кодирования}. Для дискретного стационарного канала с информационной емкостью $C_0$ для любого $\delta>0$ существует число $\varepsilon > 0$, такое, что для любого кода со скоростью $R>C_0+\delta$ средняя вероятность ошибки удовлетворяет неравенству $\overline{P_e}\ge\varepsilon$, где $\overline{P_e}=\frac{1}{N}\sum\limits_{i=1}^{N}P_{ei}$ (то есть среднее по N сообщениям).

В частности, для дискретных постоянных каналов ёмкость вычисляется по формуле $C_0=\max\limits_{\{p(x)\}}I(X;Y)$, а для двоичного симметричного канала информационная емкость вычисляется по формуле $C_0=1-\eta(p)$

\textbf{Прямая теорема кодирования}. Для дискретного постоянного канала с информационной емкостью $C_0$ для любых $\varepsilon,\delta>0$ существует достаточно большое число $n_0$ такое, что для любого натурального числа $n>n_0$ существует код длины $n$ со скоростью $R\ge C_0-\delta$, средняя вероятность ошибки которого $P_e\le\varepsilon$.

\subsection*{Непрерывные каналы дискретного времени}

Пусть модель канала задается условными плотностями распределения вида $f(\overline y|\overline x)=f(y_1,\dots,y_n|x_1,\dots,x_n)$. Пусть это канал стационарный и без памяти. Также допустим, что у нас \textit{аддитивный} шум, то есть $\overline y = \overline x + \overline z$, $f(\overline y|\overline x)=f(\overline z)$. Такая модель наызвается \textit{каналом с аддитивным шумом}.

Энергия входной последовательности -- $\sum\limits_{i=1}^{n}x_i^2$. Мощность на отсчет -- $E(\overline x)=\frac{\sum\limits_{i=1}^{n}x_i^2}{n}$. Информационная емкость непрерывного стационарного канала дискретного времени с ограничением $E$ на мощность входных сигналов -- $C_0=\sup\limits_n\max\limits_{f(\overline x):\MExpect [E(\overline x)]\le E} \frac{1}{n}I(X^n;Y^n)$. В частности, для канала без памяти $C_0=\max\limits_{f(x):\MExpect [E(x)]\le E}I(X;Y)$.

Канал называется \textit{гауссовским}, если шум в канале имеет гауссовское распределение, в общем случае многомерное. Для одномерного случая $f(z)=\frac{1}{\sqrt{2\pi N_0}}e^{-\frac{z^2}{2N_0}}$. Для такого канала информационная ёмкость $C_0=\frac{1}{2}\log(1+\frac{E}{N_0})$.
\end{document}